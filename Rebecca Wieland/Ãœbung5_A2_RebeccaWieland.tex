\documentclass[11pt]{article}

\usepackage{latexsym, float}
\usepackage[ngerman]{babel}
\usepackage[utf8]{inputenc}
\usepackage[T1]{fontenc}
\author{Rebecca Wieland}
\title{CS102 \LaTeX   Übung}
\date{1.November 2014}

\begin{document}

\maketitle

\section{Das ist der erste Abschnitt}
\textmd{Hier könnte auch anderer Text stehen.}

\section{Tabelle}

\textmd{Unsere wichtigsten Daten finden Sie in Tabelle 1.}
\begin{table}[H]
\centering
\begin{tabular}{l|c|r|r}
 & Punkte erhalten & Punkte moglich" & \% \\ \hline
Aufgabe 1 & 2 & 4 & 0.5 \\
Aufgabe 2 & 3 & 3 & 1 \\
Aufgabe 3 & 3 & 3 & 1 \\

\end{tabular}
\caption{Diese Tabelle kann auch andere Werte beinhalten.}
\label{ex:table}
\end{table}

\section{Formeln}
\subsection{Pythagoras}
\textmd{ Der Satz des Pythagoras errechnet sich wie folgt: $a^2 + b^2 = c^2$. Daraus können wir die Länge der Hypothenuse wie folgt berechnen: $c = \sqrt[]{a^2 + b^2}$}
\subsection{Summen}
\textmd{Wir können auch die Formel für eine Summe angeben:} \\
\begin{equation}
s = \sum_{t=1}^n {i} = \frac{n * (n+1)}{2}
\end{equation}
\begin{flushright}
\end{document}